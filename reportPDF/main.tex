\documentclass[12pt,a4paper]{abntex2}

% =========================
% Pacotes essenciais
% =========================
\usepackage[utf8]{inputenc}
\usepackage[T1]{fontenc}
\usepackage{lmodern}                 % fonte base
\renewcommand{\familydefault}{\sfdefault} % força sans-serif (Arial-like)
\usepackage[brazil]{babel}
\usepackage{graphicx}
\usepackage{indentfirst}
\usepackage{setspace}
\usepackage{geometry}
\usepackage{float}                   % para [H]
\usepackage{booktabs}                % tabelas mais bonitas
\usepackage{array}                   % colunas personalizadas
\usepackage{longtable}               % tabelas longas (se necessário)
\usepackage{listings}                % blocos de código / ASCII
\usepackage{caption}                 % controle fino de legendas
\usepackage{hyperref}                % links
\usepackage{enumitem}                % listas mais flexíveis

\hypersetup{
    colorlinks=true,
    linkcolor=black,
    urlcolor=blue
}

% =========================
% Configuração ABNT e layout
% =========================
\setcounter{secnumdepth}{3}
\renewcommand{\thesection}{\arabic{section}}

\geometry{
  a4paper,
  left=3cm,
  right=2cm,
  top=3cm,
  bottom=2cm
}

% Espaçamento 1,5
\OnehalfSpacing

% =========================
% Configuração de listagens (ASCII / Saídas do Java)
% =========================
\lstdefinestyle{console}{
  basicstyle=\ttfamily\small,
  columns=fullflexible,
  frame=single,
  breaklines=true,
  keepspaces=true,
  tabsize=2
}

% =========================
% Macros de placeholder em LATIM (com nome do aluno)
% =========================
\newcommand{\placeholderAluno}[1]{\textit{Lorem ipsum dolor sit amet, consectetur adipiscing elit — #1.}}
\newcommand{\placeholderAlunoLongo}[1]{\textit{Lorem ipsum dolor sit amet, consectetur adipiscing elit. Sed semper nunc est, et rutrum dui interdum quis — #1.}}

% =========================
% Título / Autores / Curso
% =========================
\title{Relatório Técnico -- Comparativo de Funções Hash em Java}
\author{
  Fernando Alonso Piroga da Silva \\
  Jafte Carneiro Fagundes da Silva \\
  Renato Pestana Gouveia
}
\date{\textbf{Pontifícia Universidade Católica do Paraná} \\ Resolução de Problemas Estruturados em Computação}

\begin{document}
\maketitle

\tableofcontents
\newpage

% =========================
% 1. INTRODUÇÃO
% =========================
\section{Introdução}
Este relatório apresenta a implementação manual de \textit{tabelas hash} em Java, comparando a eficiência de duas funções de dispersão (\textit{hash functions}) distintas, sem utilizar estruturas de dados prontas (\textit{ArrayList}, \textit{LinkedList}, \textit{HashMap}, etc.). O foco está na aplicação dos princípios de Programação Orientada a Objetos (POO), na análise de colisões, no tratamento por encadeamento, e no redimensionamento com \textit{rehashing}.

\placeholderAluno{Jafte}

\subsection{Objetivo}
Comparar o desempenho de duas funções hash de complexidade média aplicadas sobre um conjunto de 5000 nomes, medindo colisões, tempo de inserção, tempo de busca, fator de carga, distribuição das chaves e clusterização.

\placeholderAluno{Jafte}

\subsection{Escopo e Organização}
O trabalho contempla: (i) implementação base abstrata; (ii) duas implementações concretas que diferem apenas na função hash; (iii) coleta de métricas; (iv) relatório comparativo. Os resultados numéricos e gráficos ASCII apresentados nas seções seguintes serão preenchidos com a saída do programa Java.

\placeholderAluno{Jafte}

% =========================
% 2. ARQUITETURA E POO
% =========================
\section{Arquitetura e Princípios de POO}
A solução foi estruturada com ênfase em encapsulamento, abstração, herança e polimorfismo. A classe abstrata \texttt{TabelaHash} define a interface e a lógica comum, enquanto as subclasses implementam a função hash específica. O tratamento de colisões é feito via \textit{chaining} com \texttt{ListaEncadeada} e nós (\texttt{Node}) implementados manualmente.

\subsection{Diagrama de Componentes (alto nível)}
\begin{itemize}[noitemsep]
  \item \texttt{TabelaHash} (abstrata): inserir, buscar, redimensionar, métricas.
  \item \texttt{TabelaHashMetodo1}: sobrescreve \texttt{calcularHash()} -- (Fernando).
  \item \texttt{TabelaHashMetodo2}: sobrescreve \texttt{calcularHash()} -- (Renato).
  \item \texttt{ListaEncadeada} e \texttt{Node}: encadeamento manual.
  \item \texttt{LeitorArquivo}: leitura do arquivo \texttt{data/female\_names.txt}.
  \item \texttt{MedidorPerformance} e \texttt{RelatorioComparativo}: métricas/saídas.
\end{itemize}

\placeholderAlunoLongo{Jafte}

% =========================
% 3. FUNÇÕES HASH IMPLEMENTADAS
% =========================
\section{Funções Hash Implementadas}

\subsection{Função Hash 1 -- Método da Multiplicação (Fernando)}
Descrição teórica sucinta do método da multiplicação (Knuth), utilização de constante irracional \(A \approx 0.6180339887\), cálculo da parte fracionária e mapeamento para o intervalo da capacidade.

\placeholderAluno{Fernando}

\subsubsection*{Assinatura e Esboço (Java)}
\begin{lstlisting}[style=console]
protected int calcularHash(String chave, int capacidade) {
    // Método da Multiplicação (Knuth) - implementar
    // Lorem ipsum — Fernando
}
\end{lstlisting}

\subsection{Função Hash 2 -- DJB2 (Renato)}
Descrição teórica sucinta do algoritmo DJB2, motivação histórica, comportamento de dispersão e custo computacional.

\placeholderAluno{Renato}

\subsubsection*{Assinatura e Esboço (Java)}
\begin{lstlisting}[style=console]
protected int calcularHash(String chave, int capacidade) {
    // DJB2 - implementar
    // Lorem ipsum — Renato
}
\end{lstlisting}

% =========================
% 4. TRATAMENTO DE COLISÕES
% =========================
\section{Tratamento de Colisões: Encadeamento (Chaining)}
As colisões são tratadas por listas encadeadas simples implementadas manualmente. Para cada índice da tabela, mantém-se uma \texttt{ListaEncadeada} que armazena nós do tipo \texttt{Node}. Inserções em posições já ocupadas elevam o contador de colisões.

\placeholderAluno{Jafte}

% =========================
% 5. REDIMENSIONAMENTO E REHASHING
% =========================
\section{Redimensionamento e Rehashing}
O redimensionamento é disparado quando o fator de carga atinge o limite configurado (sugestão: 0{,}75). A capacidade dobra (\(32 \rightarrow 64 \rightarrow 128 \rightarrow \ldots\)) e realiza-se rehashing de todos os elementos.

\placeholderAluno{Jafte}

% =========================
% 6. CONFIGURAÇÃO DE EXECUÇÃO
% =========================
\section{Configuração de Execução}
\begin{itemize}
  \item Capacidade inicial: \textbf{32}.
  \item Fator de carga: \textbf{0,75}.
  \item Arquivo de entrada: \texttt{data/female\_names.txt} (5000 nomes).
  \item Método de colisão: \textbf{Encadeamento (Lista Encadeada manual)}.
\end{itemize}

\placeholderAluno{Jafte}

% =========================
% 7. RESULTADOS DOS TESTES (PLACEHOLDERS PARA SAÍDAS DO JAVA)
% =========================
\section{Resultados dos Testes}

\subsection{Resumo de Métricas (Tabela Comparativa)}
% ==== Cole aqui a tabela gerada no console ou preencha manualmente ====
\begin{table}[H]
\centering
\caption{Comparativo de eficiência das funções hash.}
\begin{tabular}{lcccccc}
\toprule
\textbf{Função Hash} & \textbf{Colisões} & \textbf{Redim.} & \textbf{Inserção (ms)} & \textbf{Busca (ms)} & \textbf{Fator de Carga} \\
\midrule
Multiplicação (Fernando) & \textit{Lorem ipsum — Fernando} & \textit{Lorem ipsum — Fernando} & \textit{Lorem ipsum — Fernando} & \textit{Lorem ipsum — Fernando} & \textit{Lorem ipsum — Fernando} \\
DJB2 (Renato)            & \textit{Lorem ipsum — Renato}   & \textit{Lorem ipsum — Renato}   & \textit{Lorem ipsum — Renato}   & \textit{Lorem ipsum — Renato}   & \textit{Lorem ipsum — Renato} \\
\bottomrule
\end{tabular}
\end{table}

\subsection{Distribuição das Chaves (Histograma ASCII)}
% ==== Cole aqui a saída ASCII do método mostrarDistribuicao() (ou versão resumida) ====
\begin{lstlisting}[style=console, caption={Distribuição das chaves por posição (hash 1 e/ou hash 2).}]
Lorem ipsum dolor sit amet — Jafte
\end{lstlisting}

\subsection{Análise de Clusterização (Top N Posições)}
% ==== Cole aqui as posições com maiores comprimentos de listas/colisões ====
\begin{lstlisting}[style=console, caption={Posições mais congestionadas (exemplo).}]
Lorem ipsum dolor sit amet — Jafte
\end{lstlisting}

\subsection{Tempo de Execução: Inserção e Busca}
% ==== Cole aqui o trecho do console com tempos em ms ====
\begin{lstlisting}[style=console, caption={Tempos de inserção e busca (ms).}]
Lorem ipsum dolor sit amet — Jafte
\end{lstlisting}

% =========================
% 8. DISCUSSÃO E ANÁLISE
% =========================
\section{Discussão e Análise}
\subsection{Comparação entre as Funções}
Discussão sobre dispersão, quantidade de colisões, estabilidade dos tempos, influência do redimensionamento.

\placeholderAluno{Jafte}

\subsection{Impacto do Tratamento de Colisões}
Análise do encadeamento e suas implicações em termos de complexidade amortizada, comparação com alternativas (endereçamento aberto) e justificativa de projeto.

\placeholderAluno{Jafte}

\subsection{Limitações e Ameaças à Validade}
Tamanho do dataset, variação de ambiente de execução, sensibilidade à ordem de inserção e seleção dos nomes de busca.

\placeholderAluno{Jafte}

% =========================
% 9. CONCLUSÕES
% =========================
\section{Conclusões}
Síntese dos resultados observados, indicação de qual função apresentou melhor desempenho geral e por quais razões (menos colisões, melhor distribuição, menor clusterização, tempos inferiores).

\placeholderAluno{Jafte}

% =========================
% 10. TRABALHOS FUTUROS
% =========================
\section{Trabalhos Futuros}
Evolução para outras funções hash (SDBM, FNV-1a, Polynomial Rolling), comparação com endereçamento aberto, instrumentação de variância e testes de estresse com diferentes tamanhos de entrada.

\placeholderAluno{Jafte}

% =========================
% 11. APÊNDICE (OPCIONAL): TRECHOS DE SAÍDA COMPLETA
% =========================
\section*{Apêndice A -- Saídas Completas do Console (Opcional)}
% ==== Cole aqui dumps completos do console (inserção, busca, distribuição) ====
\begin{lstlisting}[style=console]
Lorem ipsum dolor sit amet — Jafte
\end{lstlisting}

\end{document}
